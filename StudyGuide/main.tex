\documentclass[11pt]{article}
\usepackage{float}
\usepackage{fullpage}
\usepackage{amsmath}
\usepackage{amstext}
\usepackage{caption}

\title{Computer Architecture (ECS 154A) Study Guide}
\author{Davis Computer Science Club --- Tutoring Committee}
\date{Winter Quarter 2015}

\begin{document}

\maketitle

\section{Boolean Algebra}

\subsection{Operations}

\begin{table}[H]
	\centering
	\begin{tabular}{c | c | c}
		Operation	&	Symbol			&	Example\\
		\hline
		NOT			&	\( \bar{ } \)	&	\( \bar{A} \)\\
					&	\( ! \)			&	\( !A \)\\
					&	\( \neg \)		&	\( \neg A \)\\
					&	\( \sim \)		&	\( \sim A \)\\
		\hline
		AND			&	\( \wedge \)	&	\( A \wedge B \)\\
					&	\( * \)			&	\( A * B \)\\
					&					&	\( AB \)\\
		\hline
		OR			&	\( \vee \)		&	\( A \vee B \)\\
					&	\( + \)			&	\( A + B\)\\
		\hline
		XOR			&	\( \oplus \)	&	\( A \oplus B \)
	\end{tabular}
\end{table}

\subsection{Equivalence Laws}

\begin{table}[H]
	\centering
	\begin{tabular}{l | l | l}
		Name			&	OR Version										&	AND Version\\
		\hline
		Commutative		&	\( A + B  = B + A \)							&	\( A * B = B * A \)\\
		Associative		&	\( (A + B) + C = A + (B + C) \)					&	\( (A * B) * C = A * (B * C) \)\\
		Distributive	&	\( A + (B * C) = (A + B) * (A + C) \)			&	\( A * (B + C) = (A * B) + (A * C) \)\\
		Identity		&	\( A + 0 = A \)									&	\( A * 1 = A \)\\
		Annulment		&	\( A + 1 = 1 \)									&	\( A * 0 = 0 \)\\
		Idempotent		&	\( A + A = A \)									&	\( A * A = A \)\\
		Complement		&	\( A + \bar{A} = 1 \)							&	\( A * \bar{A} = 0 \)\\
		De Morgan's		&	\( \overline{(A + B)} = \bar{A} * \bar{B} \)	&	\( \overline{(A * B)} = \bar{A} + \bar{B} \)
	\end{tabular}
\end{table}

\subsection{Truth Table}

Truth tables are mathematical tables composing of every combination of inputs and the resulting function. The number of combinations or rows in the truth table is \( 2^N \), where N is the number of inputs.

\subsubsection{NOT}

\begin{table}[H]
	\centering
	\begin{tabular}{c | c}
		A	&	\( f(A) = !A \)\\
		\hline
		0	&	1\\
		1	&	0
	\end{tabular}
\end{table}

\subsubsection{AND}

\begin{table}[H]
	\centering
	\begin{tabular}{c c | c}
		A	&	B	&	\( f(A,B) = A * B \)\\
		\hline
		0	&	0	&	0\\
		0	&	1	&	0\\
		1	&	0	&	0\\
		1	&	1	&	1
	\end{tabular}
\end{table}

\subsubsection{OR}

\begin{table}[H]
	\centering
	\begin{tabular}{c c | c}
		A	&	B	&	\( f(A,B) = A + B \)\\
		\hline
		0	&	0	&	0\\
		0	&	1	&	1\\
		1	&	0	&	1\\
		1	&	1	&	1
	\end{tabular}
\end{table}

\subsubsection{XOR}

\begin{table}[H]
	\centering
	\begin{tabular}{c c | c}
		A	&	B	&	\( f(A,B) = A \oplus B \)\\
		\hline
		0	&	0	&	0\\
		0	&	1	&	1\\
		1	&	0	&	1\\
		1	&	1	&	0
	\end{tabular}
\end{table}

\subsection{Canonical Normal Form}

\subsection{Karnaugh Maps}

\subsection{Quine-McClucksey Algorithm}

\section{Combinational Logic Circuits}

\subsection{Gates}

\subsection{Timing Diagrams}

\subsection{Multiplexers, Decoders, Shifters}

\subsection{Adders and Subtracters}

\subsection{Designing Combinational Logic Circuits}

\section{Finite State Automata}

\subsection{Moore Model}

\subsection{Mealy Model}

\section{Sequential Logic Circuit}

\subsection{Latches}

\subsection{Flip Flops}

\subsection{Registers and Counters}

\subsection{Designing Sequential Logic Circuits}

\section{Single Cycle CPU Design}

\section{Cache}

Effectiveness is based on the concept of information reuse: temporal locality and spatial locality.

\subsection{Memory Hierarchy}

Goal: Make memory perform as if it was made of the most expensive and fastest type, but cost as if made of the cheapest type.

\begin{enumerate}
	\item Fast, Hot, Expensive
	\item Static RAM
	\item Dynamic RAM
	\item Disk
\end{enumerate}

\noindent A cache is smaller than main memory and is composed of numerous cache lines. Cache lines consist of a dirty bit, a tag, and block(s) of data. If the dirty bit is on, then it signals the CPU to write the data from this cache line into main memory when this cache line is freed. Caches also contain a valid bit which signifies whether there is loaded data into the cache --- imagine starting up the computer for the first time, the cache is going to be empty. The issue is that even when a cache is empty (bits are all 0), it still holds some signal. As we continue, when we mention the size of a cache line, we refer to the size of the data blocks in the cache line only (excluding flags and tag).

\begin{table}[H]
	\centering
	\begin{tabular}{| c | c | c |}
		\hline
		Flags	&	\hspace{.5cm} Tag \hspace{.5cm} 		&	\hspace{1cm} Block(s) \hspace{1cm} \\
		\hline
	\end{tabular}
\end{table}

\subsection{Direct Mapped Cache}

A given address is partitioned into three components: Tag, line number, and offset. The line number directly accesses a specific cache line. It then compares the tag partitioned from the address to the tag stored in the cache line. If the tags match, then it is a cache hit. Otherwise, it becomes a cache miss. Assuming that it was a cache hit, it then proceeds to use the offset to select which block to read or write. The block of data in the cache line can be thought of as an array and the offset as an index into this ``array".

\subsubsection{Format}

\begin{table}[H]
	\centering
	\begin{tabular}{| c | c | c |}
		\hline
		Tag		&	Line Number		&		 Offset\\
		\hline
	\end{tabular}
\end{table}

\subsubsection{Example}

\textbf{A CPU is using 24-bit addresses and is byte-addressable. Each line in cache holds 16 bytes of data and each block is 1 byte. Assuming that the tag is 12 bits wide, find the following: number of lines in cache, size of cache, size of tag, and sizes of each partition in the format.}

\[
\text{Size(Cache Line)} = 16 \text{ bytes}
\]
\begin{eqnarray*}
	\text{Bit\_Width(Offset)} &=& \text{Number of bits needed to address 16 blocks}\\
		&=& \log_2 16\\
		&=& 4 \text{ bits}
\end{eqnarray*}
\begin{eqnarray*}
\text{Bit\_Width(Line Number)} &=& \text{Size(Address)} - \text{Bit\_Width(Tag)} - \text{Bit\_Width(Offset)}\\
	&=& 24 \text{ bits} - 12 \text{ bits} - 4 \text{ bits}\\
	&=& 8 \text{ bits}
\end{eqnarray*}
\begin{eqnarray*}
	\text{Number of Cache Lines} &=& 2^\text{Bit\_Width(Line Number)}\\
		&=& 2^8\\
		&=& 256
\end{eqnarray*}
\begin{eqnarray*}
	\text{Size of Cache} &=& \text{Number of Cache Lines} * \text{Size of Cache Lines}\\
		&=& 2^8 * 2^4\\
		&=& 2048 \text{ bytes}\\
		&=& 2 \text{ KB}
\end{eqnarray*}

\begin{table}[H]
	\centering
	\caption*{Format Partition Sizes and Bit Range}
	\begin{tabular}{| c | c | c | c |}
		\hline
					&	Tag		&	Line Number		&	Offset\\
		\hline
		Size		&	12 Bits	&	8 Bits			&	4 Bits\\
		\hline
		Bit Range	&	Bits 12-23	&	Bits 4-11	&	Bits 0-3\\
		\hline
	\end{tabular}
\end{table}

\subsection{Fully Associative}

A given address is partitioned into two components: Tag and offset. The reason why a line number is unused is because the CPU will perform a linear search through all cache lines looking for either a cache hit, an unused cache line, or a cache line to replace. Because of this, using full association has the lowest miss rates. The tag and offset performs the same as the tag and offset in direct mapping.


\subsubsection{Format}

\begin{table}[H]
	\centering
	\begin{tabular}{| c | c |}
		\hline
		Tag			&		 Offset\\
		\hline
	\end{tabular}
\end{table}

\subsubsection{Example}

\textbf{A CPU is using 11-bit addresses and is byte-addressable. The cache size is 128 bytes and the size of each cache line is 8 bytes. Find the following: Bit width of offset, bit width of tag, number of cache lines, and sizes and bit ranges of the partitions.}

\begin{eqnarray*}
	\text{Bit\_Width(Offset)} &=& \text{Number of bits needed to address 8 blocks}\\
	&=& \log_2 8\\
	&=& 3 \text{ bits}
\end{eqnarray*}
\begin{eqnarray*}
	\text{Bit\_Width(Tag)} &=& \text{Size(Address)} - \text{Bit\_Width(Offset)}\\
	&=& 11 \text{ bits} - 3 \text{ bits}\\
	&=& 8 \text{ bits}
\end{eqnarray*}
\begin{eqnarray*}
	\text{Number of Cache Lines} &=& \text{Size(Cache)} / \text{Size(Cache Line)}\\
	&=& 128 \text{ bytes} / 8 \text{ bytes}\\
	&=& 16 \text{ Cache Lines}
\end{eqnarray*}

\begin{table}[H]
	\centering
	\caption*{Format Partition Sizes and Bit Range}
	\begin{tabular}{| c | c | c |}
		\hline
		&	Tag		&	Offset\\
		\hline
		Size		&	8 Bits		&	3 Bits\\
		\hline
		Bit Range	&	Bits 3-10	&	Bits 0-2\\
		\hline
	\end{tabular}
\end{table}

\subsection{Set Associative}

Set associative caching is a hybrid between direct-mapped and fully associative. A given address is partitioned into three components: Tag, set number, and offset. What makes set associative different from direct-mapped caching is that a given address will access a specific set of cache lines instead of a single cache line. Within that set of cache lines, it will perform a linear search like the fully associative cache. The tag and offset perform the same as the direct-mapped cache and fully associative cache. \textbf{Important terminology:} N-way associative cache means that there are N cache lines per set. This is also known as the associativity.

\subsubsection{Format}

\begin{table}[H]
	\centering
	\begin{tabular}{| c | c | c|}
		\hline
		Tag			&		Set Number	&	 Offset\\
		\hline
	\end{tabular}
\end{table}

\subsubsection{Example}

\textbf{A CPU is using 64-bit addresses and is byte-addressable. It also uses a 3-way set associative cache with a cache size of 98,304 bytes and 32 sets. Find the following: Number of lines per set, size of each set, size of cache line, bit width of offset, bit width of set number, bit width of tag, and partition sizes and bit ranges.}

\[
	\text{Number of Lines per Set} = 3 \text{ (Given)}
\]
\begin{eqnarray*}
	\text{Size(Set)} &=& \text{Size(Cache)} / \text{Number of Sets}\\
		&=& 98304 / 32\\
		&=& 3072 \text{ bytes}
\end{eqnarray*}
\begin{eqnarray*}
	\text{Size(Cache Line)} &=& \text{Size(Set)} / \text{ Number of lines per set}\\
	&=& 3072 \text{ bytes} / 3\\
	&=& 1024 \text{ bytes}
\end{eqnarray*}
\begin{eqnarray*}
	\text{Bit\_Width(Offset)} &=& \text{Number of bits needed to represent 1024 bytes}\\
	&=& \log_2 \text{Size(Cache Line)}\\
	&=& \log_2 1024\\
	&=& 10 \text{ bits}
\end{eqnarray*}
\begin{eqnarray*}
	\text{Bit\_Width(Set Number)} &=& \text{Number of bits needed to represent 32 sets}\\
	&=& \log_2 \text{Number of sets}\\
	&=& \log_2 32\\
	&=& 5 \text{ bits}
\end{eqnarray*}
\begin{eqnarray*}
	\text{Bit\_Width(Tag)} &=& \text{Size(Address)} - \text{Bit\_Width(Set Number)} - \text{Bit\_Width(Offset)}\\
	&=& 64 \text{ bits} - 5 \text{ bits} - 10 \text{ bits}\\
	&=& 49 \text{ bits}
\end{eqnarray*}

\begin{table}[H]
	\centering
	\caption*{Format Partition Sizes and Bit Range}
	\begin{tabular}{| c | c | c | c |}
		\hline
					&	Tag		&	Set Number		&	Offset\\
		\hline
		Size			&	12 Bits		&	5 Bits		&	10 Bits\\
		\hline
		Bit Range		&	Bits 15-63		&	Bits 10-14		&	Bits 0-9\\
		\hline
	\end{tabular}
\end{table}


\section{Virtual Memory}

\section{Multi-Cycle CPU Design}

\section{Pipeline CPU Design}

\end{document}